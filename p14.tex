\section{Pregunta N$^{\circ}$14\qquad Nombre}

\begin{frame}
	\begin{enumerate}\setcounter{enumi}{13}
		\item

		      Consideremos una población de animales hembras que tienen
		      una esperanza máxima de vida de $24$ meses.
		      Dicha población está dividida en tres grupos de edad: las
		      crías que tienen menos de $8$ meses, las jóvenes que tienen
		      al menos $8$ meses y menos de $16$ meses y las adultas que
		      tienen al menos $16$ meses y menos de $24$.
		      Se hacen recuentos periódicos de la población cada $8$ meses.
		      Se supone que todas las adultas mueren al pasar de uno a
		      otro recuento.
		      Se sabe que en cada período $\dfrac{1}{4}$ de las crías
		      llegan a jóvenes y $\dfrac{2}{3}$ de las jóvenes llegan a adultas.
		      Las crías no se reproducen, el número medio de crías
		      hembras por cada hembra joven es de $2$ y por cada hembra
		      adulta es de $3$.
		      Inicialmente hay $200$ crías, $100$ jóvenes y $80$ adultas.

		      \

		      \begin{enumerate}[a)]
			      \item

			            Modele el problema.

			            \

			      \item


			            Determine el polinomio característico usando los
			            métodos Leverrier y Krylov.

			            \

			      \item

			            Determine todos los valores y vectores propios

			            \

			      \item


			            usando los métodos dados en clase.
		      \end{enumerate}
	\end{enumerate}

	\begin{solution}
		\begin{enumerate}[a)]
			\item

			      .
		\end{enumerate}
	\end{solution}
\end{frame}