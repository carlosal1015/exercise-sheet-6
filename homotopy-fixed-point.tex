\begin{frame}
	\begin{definition}[Homotopía de punto fijo]
		Sean
		\begin{math}
			F\colon\mathbb{R}^{n}\to\mathbb{R}^{n}
		\end{math}
		y
		\begin{math}
			x\in\mathbb{R}^{n}
		\end{math}.
		Una de las homotopías más sencillas para encontrar soluciones de
		$F\left(x\right)=0$ es la \alert{homotopía de punto fijo} dada
		por
		\begin{equation*}
			H\left(x,t\right)\coloneqq
			\left(1-t\right)\left(x-a\right)+tF\left(x\right)
		\end{equation*}
		donde $a\in\mathbb{R}^{n}$ y
		$t\in\left[0,1\right]\subset\mathbb{R}$.
		Entonces, si
		\begin{columns}
			\begin{column}{0.48\textwidth}
				\begin{itemize}
					\item

					      \begin{math}
						      t=0:
						      H\left(x,0\right)=
						      x-a
					      \end{math}
					      es el sistema inicial donde la única solución es $x=a$.
				\end{itemize}
			\end{column}
			\begin{column}{0.48\textwidth}
				\begin{itemize}
					\item

					      \begin{math}
						      t=1:
						      H\left(x,1\right)=
						      F\left(x\right)=
						      0
					      \end{math}
					      es el sistema de interés.
				\end{itemize}
			\end{column}
		\end{columns}
	\end{definition}
	La suavidad de esta construcción de homotopía está garantizada por
	el siguiente:
	\begin{theorem}[Teorema de Sard]
		Sean $X\subset\mathbb{R}^{n_{1}}$ e $Y\subset\mathbb{R}^{n_{2}}$
		dos conjuntos abiertos, $\phi\colon X\times Y\to\mathbb{R}^{m}$
		suave.
		Si $v\in\mathbb{R}^{m}$ es un punto regular de $\phi$, entonces
		para casi todo $y\in Y$, $v$ es un punto regular de
		$\pi_{Y}\colon X\to\mathbb{R}^{m}$.
	\end{theorem}
\end{frame}

\begin{frame}

	\begin{theorem}[Existencia de soluciones de un sistema no lineal]
		Sean
		\begin{math}
			F\colon\mathbb{R}^{n}\to\mathbb{R}^{n}
		\end{math}
		suave y $U\subset\mathbb{R}^{n}$ un conjunto compacto con
		interior no vacío.
		Si para cualquier $x\in\partial U$, ${F\left(x\right)}^{T}\left(x-a\right)>0$,
		entonces existe un $x^{\ast}\in\operatorname{int}\left(U\right)$ tal que
		$F\left(x^{\ast}\right)=0$.
	\end{theorem}

	\begin{proof}
		Sea
		\begin{math}
			H\colon
			\mathbb{R}^{n}\times\left[0,1\right]\to
			\mathbb{R}^{n}
		\end{math}
		una homotopía definida como
		\begin{math}
			H\left(x,t\right)\coloneqq
			\left(1-t\right)\left(x-a\right)+
			tF\left(x\right).
		\end{math}
		Como antes, para casi todo $a$, $H$ satisface la condición de
		suavidad para $t\neq 1$.
		Esto garantiza que el conjunto solución de $H\left(x,t\right)=0$
		consiste de todas las curvas suaves.
		Por simple inspección, $a$ es la única solución de $H\left(x,t\right)=0$.
		También, si $x\in\partial U$, entonces
		\begin{equation*}
			{H\left(x,t\right)}^{T}\left(x-a\right)=
			\left(1-t\right){\left\|x-a\right\|}^{2}+
			t{F\left(x\right)}^{T}\left(x-a\right)>0.
		\end{equation*}
		Entonces, la curva solución definida por $H\left(x,t\right)=0$
		que emana de $\left(a,0\right)$ no puede volver a $t=0$, y no
		puede alcanzar la frontera de $U$.
		Por lo tanto, debe extenderse a $t=1$ dentro del interior de $U$,
		dando un punto
		\begin{math}
			\left(x^{\ast},1\right)\in
			H^{-1}\left(\left\{0\right\}\right)
		\end{math}
		con $x^{\ast}\in U$ que satisface $F\left(x^{\ast}\right)=0$.
	\end{proof}
\end{frame}