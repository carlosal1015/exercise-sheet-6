\section{Pregunta N$^{\circ}$11\qquad Alejandro Escobar Mejia}

\begin{frame}
	\begin{enumerate}\setcounter{enumi}{10}
		\item

		      Un territorio está dividido en dos zonas $Z_{1}$ y $Z_{2}$
		      entre las que habita una población de aves.
		      Cada año y debido a diversas razones (disponibilidad de
		      alimentos, peleas por el territorio, etc.) se producen
		      las siguientes flujos migratorios entre las distintas
		      zonas.

		      \begin{columns}
			      \begin{column}{0.48\textwidth}
				      \begin{itemize}
					      \item

					            En $Z_{1}$: un $60$\% permanece en $Z_{1}$ y un
					            $40$\% migra a $Z_{2}$.
				      \end{itemize}
			      \end{column}
			      \begin{column}{0.48\textwidth}
				      \begin{itemize}
					      \item

					            En $Z_{2}$: un $20$\% migra a $Z_{1}$ y un
					            $80$\% permanece en $Z_{2}$.
				      \end{itemize}
			      \end{column}
		      \end{columns}

		      Supongamos que tenemos una situación inicial en la que la
		      población total de aves un $60$\% viven en $Z_{1}$, un
		      $40$\% viven en $Z_{2}$.

		      \

		      \begin{enumerate}[a)]
			      \item

			            Indique las variables a usar.

			            \

			      \item

			            Determine la matriz que define la migración.

			            \

			      \item

			            Determine el polinomio característico usando los
			            métodos Leverrier y Krylov.

			            \

			      \item

			            Determine todos los valores y vectores propios
			            usando los métodos dados en clase.
		      \end{enumerate}
	\end{enumerate}

	\begin{solution}
		\begin{enumerate}[a)]
			\item

			      .
		\end{enumerate}
	\end{solution}
\end{frame}