\section{Pregunta N$^{\circ}$11\qquad Alejandro Escobar Mejia}

\begin{frame}
	\begin{enumerate}\setcounter{enumi}{10}
		\item

		      Un territorio está dividido en dos zonas $Z_{1}$ y
		      $Z_{2}$ entre las que habita una población de aves.
		      Cada año y debido a diversas razones (disponibilidad de
		      alimentos, peleas por el territorio, etc.) se producen
		      las siguientes flujos migratorios entre las distintas
		      zonas.

		      \begin{columns}
			      \begin{column}{0.48\textwidth}
				      \begin{itemize}
					      \item

					            En $Z_{1}$: un $60$\% permanece en $Z_{1}$
					            y un $40$\% migra a $Z_{2}$.
				      \end{itemize}
			      \end{column}
			      \begin{column}{0.48\textwidth}
				      \begin{itemize}
					      \item

					            En $Z_{2}$: un $20$\% migra a $Z_{1}$ y un
					            $80$\% permanece en $Z_{2}$.
				      \end{itemize}
			      \end{column}
		      \end{columns}

		      Supongamos que tenemos una situación inicial en la que la
		      población total de aves un $60$\% viven en $Z_{1}$, un
		      $40$\% viven en $Z_{2}$.

		      \

		      \begin{enumerate}[a)]
			      \item

			            Indique las variables a usar.

			            \

			      \item

			            Determine la matriz que define la migración.

			            \

			      \item

			            Determine el polinomio característico usando los
			            métodos Leverrier y Krylov.

			            \

			      \item

			            Determine todos los valores y vectores propios
			            usando los métodos dados en clase.
		      \end{enumerate}
	\end{enumerate}

	\begin{solution}
		\begin{enumerate}[a)]
			\item

			      Sean $x_{1}$ y $x_{2}$ la cantidad de aves en la zona
			      $Z_{1}$ y $Z_{2}$, respectivamente.

			\item
			      \begin{align*}
				      x^{\left(k+1\right)}_{1} & =
				      0.6x^{\left(k\right)}_{1}+
				      0.2x_{2}^{\left(k\right)}    \\
				      x^{\left(k+1\right)}_{2} & =
				      0.4x^{\left(k\right)}_{1}+
				      0.8x^{\left(k\right)}_{2}
			      \end{align*}
		\end{enumerate}

		\begin{equation*}
			\begin{pmatrix}
				x^{\left(k+1\right)}_{1} \\
				x^{\left(k+1\right)}_{2}
			\end{pmatrix}
			=
			\begin{pmatrix}
				0.6 & 0.2 \\
				0.4 & 0.8
			\end{pmatrix}
			\begin{pmatrix}
				x^{\left(k\right)}_{1} \\
				x^{\left(k\right)}_{2}
			\end{pmatrix}
		\end{equation*}

		\begin{equation*}
			x^{\left(k+1\right)}=
			Ax^{\left(k\right)}
		\end{equation*}
	\end{solution}
\end{frame}

\begin{frame}
	\begin{solution}
		\begin{enumerate}[c)]
			\item

			      Método de Leverrier:

			      \begin{equation*}
				      A=\begin{pmatrix}
					      0.6 & 0.2 \\
					      0.4 & 0.8
				      \end{pmatrix}
			      \end{equation*}

			      \begin{equation*}
				      A^{2}=
				      \begin{pmatrix}
					      0.44 & 0.28 \\
					      0.56 & 0.72
				      \end{pmatrix}
			      \end{equation*}

			      \begin{math}
				      s_{1}=\operatorname{traz}\left(A\right)=1.4\\
				      s_{2}=\operatorname{traz}\left(A^{2}\right)=1.16\\
				      a_{1}=-s_{1}=-1.4\\
				      a_{2}=-\frac{1}{2}\left(s_{2}+(a_{1}\times s_{1}\right)\right)=
				      -\frac{1}{2}\left(1.16+\left(-1.4\times 1.4\right)\right)=0.4
			      \end{math}
			      \begin{equation*}
				      P\left(x\right)=
				      x^{2}-1.4x+0.4
			      \end{equation*}
			      .
		\end{enumerate}
	\end{solution}
\end{frame}

\begin{frame}
	\begin{solution}
		Método de Krylov:

		\begin{equation*}
			y_{0}=
			\begin{pmatrix}
				1 \\
				0
			\end{pmatrix}
		\end{equation*}

		\begin{equation*}
			y_{1}=
			Ay_{0}=
			\begin{pmatrix}
				0.6 \\
				0.4
			\end{pmatrix}
		\end{equation*}

		\begin{equation*}
			y_{2}=
			Ay_{1}=
			\begin{pmatrix}
				0.44 \\
				0.56
			\end{pmatrix}
		\end{equation*}

		\begin{equation*}
			\begin{pmatrix}
				y_{1} & y_{0}
			\end{pmatrix}
			\begin{pmatrix}
				a_{1} \\
				a_{2}
			\end{pmatrix}=
			y_{2}
		\end{equation*}

		\begin{equation*}
			\begin{pmatrix}
				0.6 & 1 \\
				0.4 & 0
			\end{pmatrix}
			\begin{pmatrix}
				a_{1} \\
				a_{2}
			\end{pmatrix}=
			\begin{pmatrix}
				0.44 \\
				0.56
			\end{pmatrix}
		\end{equation*}

		\begin{equation*}
			\begin{pmatrix}
				a_{1} \\
				a_{2}
			\end{pmatrix}=
			\begin{pmatrix}
				-1.4 \\
				0.4
			\end{pmatrix}
		\end{equation*}

		\begin{equation*}
			P\left(x\right)=
			x^{2}-1.4x+0.4
		\end{equation*}
	\end{solution}
\end{frame}

\begin{frame}

	\begin{solution}
		\begin{enumerate}[d)]
			\item

			      Método de Potencia
			      \begin{equation*}
				      A=
				      \begin{pmatrix}
					      0.6 & 0.2 \\
					      0.4 & 0.8
				      \end{pmatrix}
			      \end{equation*}

			      \begin{equation*}
				      x_{0}=
				      \begin{pmatrix}
					      1 \\
					      0
				      \end{pmatrix}
			      \end{equation*}

			      \begin{equation*}
				      x_{1}=
				      Ax_{0}=
				      \begin{pmatrix}
					      0.6 \\
					      0.4
				      \end{pmatrix}\to
				      0.4
				      \begin{pmatrix}
					      1.5 \\
					      1
				      \end{pmatrix}
			      \end{equation*}

			      \begin{equation*}
				      x_{2}=
				      Ax_{1}=
				      \begin{pmatrix}
					      0.44 \\
					      0.56
				      \end{pmatrix}\to
				      0.4
				      \begin{pmatrix}
					      1 \\
					      1.27
				      \end{pmatrix}
			      \end{equation*}

			      \begin{equation*}
				      x_{6}=
				      Ax_{5}=
				      \begin{pmatrix}
					      0.336064 \\
					      0.663936
				      \end{pmatrix}\to
				      0.336064
				      \begin{pmatrix}
					      1 \\
					      1.975623
				      \end{pmatrix}\approx
				      \begin{pmatrix}
					      1 \\
					      2
				      \end{pmatrix}
			      \end{equation*}
			      Entonces podemos ver que el vector $x_{6}$ se acerca
			      al autovector dominante de autovalor $1$.
		\end{enumerate}
	\end{solution}

\end{frame}
\begin{frame}
	\begin{solution}
		Método del Potencia Escalado
		\begin{equation*}
			A=
			\begin{pmatrix}
				0.6 & 0.2 \\
				0.4 & 0.8
			\end{pmatrix}
		\end{equation*}

		\begin{equation*}
			x_{0}=
			\begin{pmatrix}
				1 \\
				0
			\end{pmatrix}
		\end{equation*}
		\begin{equation*}
			Ax_{0}=
			\begin{pmatrix}
				0.6 \\
				0.4
			\end{pmatrix}\to
			x_{1}=
			\begin{pmatrix}
				1 \\
				0.66
			\end{pmatrix}
		\end{equation*}

		\begin{equation*}
			Ax_{1}=
			\begin{pmatrix}
				0.73 \\
				0.93
			\end{pmatrix}\to
			x_{2}=
			\begin{pmatrix}
				0.785 \\
				1
			\end{pmatrix}
		\end{equation*}

		\begin{equation*}
			Ax_{5}=
			\begin{pmatrix}
				0.73 \\
				0.93
			\end{pmatrix}\to
			x_{6}=
			\begin{pmatrix}
				0.506 \\
				1
			\end{pmatrix}
		\end{equation*}
		Entonces podemos ver que el vector $x_{6}$ se acerca al
		autovector dominante de autovalor $1$.
	\end{solution}
\end{frame}