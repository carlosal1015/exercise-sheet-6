\section{Pregunta N$^{\circ}$11\qquad Alejandro Escobar Mejia}

\begin{frame}
	\begin{enumerate}\setcounter{enumi}{10}
		\item

		      Un territorio está dividido en dos zonas $Z_{1}$ y
		      $Z_{2}$ entre las que habita una población de aves.
		      Cada año y debido a diversas razones (disponibilidad de
		      alimentos, peleas por el territorio, etc.) se producen
		      las siguientes flujos migratorios entre las distintas
		      zonas.

		      \begin{columns}
			      \begin{column}{0.48\textwidth}
				      \begin{itemize}
					      \item

					            En $Z_{1}$: un $60$\% permanece en $Z_{1}$
					            y un $40$\% migra a $Z_{2}$.
				      \end{itemize}
			      \end{column}
			      \begin{column}{0.48\textwidth}
				      \begin{itemize}
					      \item

					            En $Z_{2}$: un $20$\% migra a $Z_{1}$ y un
					            $80$\% permanece en $Z_{2}$.
				      \end{itemize}
			      \end{column}
		      \end{columns}

		      Supongamos que tenemos una situación inicial en la que la
		      población total de aves un $60$\% viven en $Z_{1}$, un
		      $40$\% viven en $Z_{2}$.

		      \begin{enumerate}[a)]
			      \item

			            Indique las variables a usar.

			      \item

			            Determine la matriz que define la migración.

			      \item

			            Determine el polinomio característico usando los
			            métodos Leverrier y Krylov.

			      \item

			            Determine todos los valores y vectores propios
			            usando los métodos dados en clase.
		      \end{enumerate}
	\end{enumerate}

	\begin{solution}
		\begin{enumerate}[a)]
			\item

			      Sean $x_{1}$ y $x_{2}$ la cantidad de aves en la zona
			      $Z_{1}$ y $Z_{2}$, respectivamente.

			\item

			      \begin{equation*}
				      \begin{aligned}
					      x^{\left(k+1\right)}_{1} & =
					      0.6x^{\left(k\right)}_{1}+
					      0.2x_{2}^{\left(k\right)}    \\
					      x^{\left(k+1\right)}_{2} & =
					      0.4x^{\left(k\right)}_{1}+
					      0.8x^{\left(k\right)}_{2}
				      \end{aligned}\iff
				      \begin{bNiceMatrix}
					      x^{\left(k+1\right)}_{1} \\
					      x^{\left(k+1\right)}_{2}
				      \end{bNiceMatrix}
				      =
				      \begin{bNiceMatrix}
					      0.6 & 0.2 \\
					      0.4 & 0.8
				      \end{bNiceMatrix}
				      \begin{bNiceMatrix}
					      x^{\left(k\right)}_{1} \\
					      x^{\left(k\right)}_{2}
				      \end{bNiceMatrix}\iff
				      x^{\left(k+1\right)}=
				      Ax^{\left(k\right)}.
			      \end{equation*}

			\item

			      Método de Leverrier

			      \begin{equation*}
				      A=\begin{bNiceMatrix}
					      0.6 & 0.2 \\
					      0.4 & 0.8
				      \end{bNiceMatrix},\quad
				      A^{2}=
				      \begin{bNiceMatrix}
					      0.44 & 0.28 \\
					      0.56 & 0.72
				      \end{bNiceMatrix},\quad
				      \begin{aligned}
					      s_{1} & =
					      \operatorname{traz}
					      \left(A\right)=
					      0.6+0.8=
					      1.4.      \\
					      s_{2} & =
					      \operatorname{traz}
					      \left(A^{2}\right)=
					      0.44+0.72=
					      1.16.     \\
					      a_{1} & =
					      -s_{1}=
					      -1.4.     \\
					      a_{2} & =
					      -\frac{1}{2}
					      \left(
					      s_{2}+
					      \left(a_{1}\times s_{1}\right)
					      \right)=
					      -\frac{1}{2}
					      \left(
					      1.16+
					      \left(-1.4\times 1.4\right)
					      \right)=
					      0.4.
				      \end{aligned}
			      \end{equation*}
		\end{enumerate}
	\end{solution}
\end{frame}

\begin{frame}
	\begin{solution}
		\begin{enumerate}[c)]
			\item

			      Así,
			      \begin{math}
				      P\left(x\right)=
				      x^{2}-1.4x+0.4
			      \end{math}.

			      Método de Krylov:

			      \begin{equation*}
				      \begin{aligned}
					      y_{0} & =
					      \begin{bNiceMatrix}
						      1 \\
						      0
					      \end{bNiceMatrix}. \\
					      y_{1} & =
					      Ay_{0}=
					      \begin{bNiceMatrix}
						      0.6 \\
						      0.4
					      \end{bNiceMatrix}. \\
					      y_{2} & =
					      Ay_{1}=
					      \begin{bNiceMatrix}
						      0.44 \\
						      0.56
					      \end{bNiceMatrix}.
				      \end{aligned},\quad
				      \begin{aligned}
					      \begin{bNiceMatrix}
						      y_{1} & y_{0}
					      \end{bNiceMatrix}
					      \begin{bNiceMatrix}
						      a_{1} \\
						      a_{2}
					      \end{bNiceMatrix} & =
					      y_{2}.                \\
					      \begin{bNiceMatrix}
						      0.6 & 1 \\
						      0.4 & 0
					      \end{bNiceMatrix}
					      \begin{bNiceMatrix}
						      a_{1} \\
						      a_{2}
					      \end{bNiceMatrix} & =
					      \begin{bNiceMatrix}
						      0.44 \\
						      0.56
					      \end{bNiceMatrix}.    \\
					      \begin{bNiceMatrix}
						      a_{1} \\
						      a_{2}
					      \end{bNiceMatrix} & =
					      \begin{bNiceMatrix}
						      -1.4 \\
						      0.4
					      \end{bNiceMatrix}.
				      \end{aligned}
			      \end{equation*}

			      Así,
			      \begin{math}
				      P\left(x\right)=
				      x^{2}-1.4x+0.4
			      \end{math}.
		\end{enumerate}

		\begin{enumerate}[d)]
			\item

			      Método de Potencia
			      \begin{equation*}
				      A=
				      \begin{bNiceMatrix}
					      0.6 & 0.2 \\
					      0.4 & 0.8
				      \end{bNiceMatrix},\quad
				      x_{0}=
				      \begin{bNiceMatrix}
					      1 \\
					      0
				      \end{bNiceMatrix},\quad
				      \begin{aligned}
					      x_{1} & =
					      Ax_{0}=
					      \begin{bNiceMatrix}
						      0.6 \\
						      0.4
					      \end{bNiceMatrix}\to
					      0.4
					      \begin{bNiceMatrix}
						      1.5 \\
						      1
					      \end{bNiceMatrix} \\
					      x_{2} & =
					      Ax_{1}=
					      \begin{bNiceMatrix}
						      0.44 \\
						      0.56
					      \end{bNiceMatrix}\to
					      0.4
					      \begin{bNiceMatrix}
						      1 \\
						      1.27
					      \end{bNiceMatrix} \\
					      x_{6} & =
					      Ax_{5}=
					      \begin{bNiceMatrix}
						      0.336064 \\
						      0.663936
					      \end{bNiceMatrix}\to
					      0.336064
					      \begin{bNiceMatrix}
						      1 \\
						      1.975623
					      \end{bNiceMatrix}\approx
					      \begin{bNiceMatrix}
						      1 \\
						      2
					      \end{bNiceMatrix}.
				      \end{aligned}
			      \end{equation*}

			      Entonces podemos ver que el vector $x_{6}$ se acerca
			      al autovector dominante de autovalor $1$.
		\end{enumerate}
	\end{solution}
\end{frame}

\begin{frame}
	\begin{solution}
		\begin{enumerate}[d)]
			\item

			      Método de Potencia Escalado
			      \begin{equation*}
				      A=
				      \begin{bNiceMatrix}
					      0.6 & 0.2 \\
					      0.4 & 0.8
				      \end{bNiceMatrix},\quad
				      x_{0}=
				      \begin{bNiceMatrix}
					      1 \\
					      0
				      \end{bNiceMatrix},\quad
				      \begin{aligned}
					      Ax_{0} & =
					      \begin{bNiceMatrix}
						      0.6 \\
						      0.4
					      \end{bNiceMatrix}\to
					      x_{1}=
					      \begin{bNiceMatrix}
						      1 \\
						      0.66
					      \end{bNiceMatrix} \\
					      Ax_{1} & =
					      \begin{bNiceMatrix}
						      0.73 \\
						      0.93
					      \end{bNiceMatrix}\to
					      x_{2}=
					      \begin{bNiceMatrix}
						      0.785 \\
						      1
					      \end{bNiceMatrix} \\
					      Ax_{5} & =
					      \begin{bNiceMatrix}
						      0.73 \\
						      0.93
					      \end{bNiceMatrix}\to
					      x_{6}=
					      \begin{bNiceMatrix}
						      0.506 \\
						      1
					      \end{bNiceMatrix}
				      \end{aligned}
			      \end{equation*}

			      Entonces podemos ver que el vector $x_{6}$ se
			      acerca al autovector dominante de autovalor $1$.
		\end{enumerate}
	\end{solution}
\end{frame}