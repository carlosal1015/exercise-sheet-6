\section{Pregunta N$^{\circ}$11\qquad Alejandro Escobar Mejia}

\begin{frame}
	\begin{enumerate}\setcounter{enumi}{10}
		\item

		      Un territorio está dividido en dos zonas $Z_{1}$ y $Z_{2}$
		      entre las que habita una población de aves.
		      Cada año y debido a diversas razones (disponibilidad de
		      alimentos, peleas por el territorio, etc.) se producen
		      las siguientes flujos migratorios entre las distintas
		      zonas.

		      \begin{columns}
			      \begin{column}{0.48\textwidth}
				      \begin{itemize}
					      \item

					            En $Z_{1}$: un $60$\% permanece en $Z_{1}$ y un
					            $40$\% migra a $Z_{2}$.
				      \end{itemize}
			      \end{column}
			      \begin{column}{0.48\textwidth}
				      \begin{itemize}
					      \item

					            En $Z_{2}$: un $20$\% migra a $Z_{1}$ y un
					            $80$\% permanece en $Z_{2}$.
				      \end{itemize}
			      \end{column}
		      \end{columns}

		      Supongamos que tenemos una situación inicial en la que la
		      población total de aves un $60$\% viven en $Z_{1}$, un
		      $40$\% viven en $Z_{2}$.

		      \

		      \begin{enumerate}[a)]
			      \item

			            Indique las variables a usar.

			            \

			      \item

			            Determine la matriz que define la migración.

			            \

			      \item

			            Determine el polinomio característico usando los
			            métodos Leverrier y Krylov.

			            \

			      \item

			            Determine todos los valores y vectores propios
			            usando los métodos dados en clase.
		      \end{enumerate}
	\end{enumerate}

	\begin{solution}
		\begin{enumerate}[a)]
			\item
            Para este problema usaremos las variables $x_{1}$ y $x_{2}$, donde :\\
            $x_{1}$: Cantidad de Aves en la zona $Z_{1}$\\
            $x_{2}$: Cantidad de Aves en la zona $Z_{2}$    
		\end{enumerate}
   \begin{enumerate}[b)]
			\item
         \begin{equation*}
            x_{1}^{k+1}=0.6x_{1}^{k}+0.2x_{2}^{k}
         \end{equation*}
         \begin{equation*}
            x_{2}^{k+1}=0.4x_{1}^{k}+0.8x_{2}^{k}
         \end{equation*}
		\end{enumerate}
  
   
	\end{solution}

\end{frame}
 \begin{frame}
  \begin{solution}
        \begin{equation*}
            \begin{pmatrix}x_{1}^{k+1}\\\\x_{2}^{k+1}\end{pmatrix}
            =
            \begin{pmatrix}0.6 & 0.2\\\\0.4 & 0.8 \end{pmatrix}
            \begin{pmatrix}x_{1}^{k}\\\\x_{2}^{k}\end{pmatrix}
         \end{equation*}
         \\
         \begin{equation*}
            x^{k+1}=Ax^{k}
         \end{equation*}
      \begin{enumerate}[c)]
			\item
            Método de Leverrier:\\
            \begin{equation*}
            A=\begin{pmatrix}0.6 & 0.2\\0.4 & 0.8 \end{pmatrix}
            
         \end{equation*}
         \begin{equation*}
            A^{2}=\begin{pmatrix}0.44 & 0.28\\0.56 & 0.72 \end{pmatrix}
         \end{equation*}
         \begin{math}
        s_{1}=Traz(A)=1.4\\
            s_{2}=Traz(A^{2})=1.16\\
            a_{1}=-s_{1}=-1.4\\
            a_{2}=-\frac{1}{2}(s_{2}+(a_{1}*s_{1})=-\frac{1}{2}(1.16+(-1.4*1.4)=0.4\\
         \end{math}
        \begin{equation*}
            P(x)=x^{2}-1.4x+0.4
         \end{equation*}
			      .
		\end{enumerate}
   
  \end{solution}
  \end{frame}
  
  \begin{frame}
  \begin{solution}
  Método de Krylov:\\
        \begin{equation*}
           y_{0}= \begin{pmatrix}1\\0 \end{pmatrix}
         \end{equation*}
         \begin{equation*}
           y_{1}=Ay_{0}= \begin{pmatrix}0.6\\0.4 \end{pmatrix}
         \end{equation*}
         \begin{equation*}
           y_{2}=Ay_{1}= \begin{pmatrix}0.44\\0.56 \end{pmatrix}
         \end{equation*}
         \begin{equation*}
            \begin{pmatrix}y_{1} & y_{0} \end{pmatrix}\begin{pmatrix}a_{1}\\a_{2} \end{pmatrix}=y_{2}
         \end{equation*}
         \begin{equation*}
            \begin{pmatrix}0.6 & 1\\0.4 &  0\end{pmatrix}\begin{pmatrix}a_{1}\\a_{2} \end{pmatrix}=\begin{pmatrix}0.44\\0.56 \end{pmatrix}
         \end{equation*}
         \begin{equation*}
            \begin{pmatrix}a_{1}\\a_{2} \end{pmatrix}=\begin{pmatrix}-1.4\\0.4 \end{pmatrix}
         \end{equation*}
         \begin{equation*}
            P(x)=x^{2}-1.4x+0.4
         \end{equation*}
  \end{solution}    
\end{frame}

\begin{frame}
    
\begin{solution}
     \begin{enumerate}[d)]
			\item
            Método de Potencia:
            \begin{equation*}
            A=\begin{pmatrix}0.6 & 0.2\\0.4 & 0.8 \end{pmatrix}   
         \end{equation*}
        \begin{equation*}
           x_{0}= \begin{pmatrix}1\\0 \end{pmatrix}
         \end{equation*}
         \begin{equation*}
           x_{1}=Ax_{0}= \begin{pmatrix}0.6\\0.4 \end{pmatrix}->0.4\begin{pmatrix}1.5\\1 \end{pmatrix}
         \end{equation*}
         \begin{equation*}
           x_{2}=Ax_{1}= \begin{pmatrix}0.44\\0.56 \end{pmatrix}->0.4\begin{pmatrix}1\\1.27 \end{pmatrix}
         \end{equation*}
         
         \begin{equation*}
           x_{6}=Ax_{5}= \begin{pmatrix}0.336064\\0.663936 \end{pmatrix}->0.336064\begin{pmatrix}1\\1.975623 \end{pmatrix} \approx \begin{pmatrix}1\\2 \end{pmatrix}
         \end{equation*}
		Entonces podemos ver que el vector $x_{6}$ se acerca a el autovector dominante de autovalor 1 
		\end{enumerate}
\end{solution}

\end{frame}
  \begin{frame}
\begin{solution}
    Método del Potencia Escalado
    \begin{equation*}
            A=\begin{pmatrix}0.6 & 0.2\\0.4 & 0.8 \end{pmatrix}   
         \end{equation*}
          \begin{equation*}
           x_{0}= \begin{pmatrix}1\\0 \end{pmatrix}
         \end{equation*}
         \begin{equation*}
           Ax_{0}= \begin{pmatrix}0.6\\0.4 \end{pmatrix}->x_{1}=\begin{pmatrix}1\\0.66 \end{pmatrix}
         \end{equation*}
         \begin{equation*}
           Ax_{1}= \begin{pmatrix}0.73\\0.93 \end{pmatrix}->x_{2}=\begin{pmatrix}0.785\\1 \end{pmatrix}
         \end{equation*}
         \begin{equation*}
           Ax_{5}= \begin{pmatrix}0.73\\0.93 \end{pmatrix}->x_{6}=\begin{pmatrix}0.506\\1 \end{pmatrix}
         \end{equation*}
         Entonces podemos ver que el vector $x_{6}$ se acerca a el autovector dominante de autovalor 1 
\end{solution}
    
\end{frame}