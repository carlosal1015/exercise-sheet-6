\begin{frame}
	Una función de homotopía comúnmente utilizada es
	\begin{definition}[Homotopía de Newton]
		La \alert{homotopía de Newton}
		$H\colon\mathbb{R}^{n}\times\mathbb{R}\to\mathbb{R}^{n}$ viene
		dada por
		\begin{equation}\label{eq:newtonhomotopy}
			H\left(x,t\right)\coloneqq
			\left(1-t\right)
			\left[
				F\left(x\right)-
				F\left(a\right)
				\right]+
			tF\left(x\right)
		\end{equation}
		donde
		\begin{math}
			F\colon\mathbb{R}^{n}\to\mathbb{R}^{n}
		\end{math}
		es el sistema suave de interés y $a\in\mathbb{R}^{n}$ es un punto
		elegido genéricamente.
		Entonces, si
	\end{definition}

	\begin{columns}
		\begin{column}{0.28\textwidth}
			\begin{itemize}
				\item

				      \begin{math}
					      t=1:
					      H\left(x,1\right)=
					      F\left(x\right)=
					      0.
				      \end{math}
			\end{itemize}
		\end{column}
		\begin{column}{0.68\textwidth}
			\begin{itemize}
				\item

				      \begin{math}
					      t=0:
					      H\left(x,0\right)=
					      F\left(x\right)-F\left(a\right)=
					      0
				      \end{math}
				      es el sistema inicial donde $a$ es solución.
			\end{itemize}
		\end{column}
	\end{columns}

	Si $H$ satisface la condición de suavidad, entonces el
	conjunto solución de $H\left(x,t\right)=0\in\mathbb{R}^{n+1}$.

	\begin{alertblock}{Conexión entre la homotopía de Newton y el método de Newton para resolver un sistema no lineal}
		Derivando ambos lados de $H\left(x,t\right)=0$ dado
		por~\eqref{eq:newtonhomotopy} se obtiene el problema de valor
		inicial
		\begin{equation*}
			\left\{
			\begin{aligned}
				\dot{x}\left(t\right) & =
				-{\left(DF\left(x\left(t\right)\right)\right)}^{-1}
				F\left(a\right),\quad t\in\left[0,1\right] \\
				x\left(0\right)       & =a
			\end{aligned}
			\right.
		\end{equation*}
		en dominios en los que $DF\left(x\right)$ es no singular.
		Aplicando el \alert{método de Euler} en $t=0$ con tamaño de paso
		$1$ a la EDO anterior desde el punto inicial $x=a$, la aproximación
		de $x\left(1\right)$ se convierte en
		\begin{equation*}
			x\left(1\right)=
			a-{\left(DF\left(a\right)\right)}^{-1}F\left(a\right)
		\end{equation*}
		que es precisamente una única iteración del método de Newton.
		Por lo tanto, la iteración de Newton puede considerarse como la
		aplicación del método de Euler con tamaño de paso $1$ en la curva
		solución dada por la homotopía de Newton~\eqref{eq:newtonhomotopy}.
	\end{alertblock}
\end{frame}