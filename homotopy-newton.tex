\begin{frame}
	Una función de homotopía comúnmente utilizada es
	\begin{definition}[Homotopía de Newton]
		Sean
		\begin{math}
			F\colon\mathbb{R}^{n}\to\mathbb{R}^{n}
		\end{math}
		suave y $a\in\mathbb{R}^{n}$.
		La \alert{homotopía de Newton}
		\begin{math}
			H\colon
			\left[0,1\right]\times\mathbb{R}^{n}\to
			\mathbb{R}^{n}
		\end{math}
		es definida como
		\begin{equation}\label{eq:newtonhomotopy}
			H\left(t,x\right)\coloneqq
			\left(1-t\right)
			\left[
				F\left(x\right)-
				F\left(a\right)
				\right]+
			tF\left(x\right)
		\end{equation}
		Entonces, si
	\end{definition}

	\begin{columns}
		\begin{column}{0.28\textwidth}
			\begin{itemize}
				\item

				      \begin{math}
					      t=1:
					      H\left(x,1\right)=
					      F\left(x\right)=
					      0.
				      \end{math}
			\end{itemize}
		\end{column}
		\begin{column}{0.68\textwidth}
			\begin{itemize}
				\item

				      \begin{math}
					      t=0:
					      H\left(x,0\right)=
					      F\left(x\right)-F\left(a\right)=
					      0
				      \end{math}
				      es el sistema inicial donde $a$ es solución.
			\end{itemize}
		\end{column}
	\end{columns}

	Si $H$ satisface la condición de suavidad, entonces el
	conjunto solución de $H\left(t,x\right)=0\in\mathbb{R}^{n}$.

	\begin{alertblock}{
			Conexión entre la homotopía de Newton y el método de Newton
			para resolver un sistema no lineal
		}
		Derivando ambos lados de
		\begin{equation*}
			H\left(t,x\left(t\right)\right)=
			\left(1-t\right)
			\left[
				F\left(x\right)-
				F\left(a\right)
				\right]+
			tF\left(x\right)=
			F\left(x\right)-
			F\left(a\right)-
			tF\left(x\right)+
			tF\left(a\right)+
			tF\left(x\right)=
			F\left(x\left(t\right)\right)+
			\left(t-1\right)
			F\left(a\right)=0.
		\end{equation*}
		dado por~\eqref{eq:newtonhomotopy} se obtiene
		\begin{align*}
			H_{t}\left(t,x\left(t\right)\right)+
			H_{x}\left(t,x\left(t\right)\right)
			\dot{x}\left(t\right) & =0           \\
			H_{x}\left(t,x\left(t\right)\right)
			\dot{x}\left(t\right) & =
			-H_{t}\left(t,x\left(t\right)\right) \\
			\dot{x}\left(t\right) & =
			-{\left(H_{x}\left(t,x\left(t\right)\right)\right)}^{-1}
			H_{t}\left(t,x\left(t\right)\right)
		\end{align*}
		Pero, $F\left(x\left(0\right)\right)=F\left(a\right)$ sii
		$x\left(0\right)=a$.
	\end{alertblock}
\end{frame}

\begin{frame}
	el problema de valor
	inicial
	\begin{equation}\label{eq:ivp}
		\left\{
		\begin{aligned}
			\dot{x}\left(t\right) & =
			-{\left(DF\left(x\left(t\right)\right)\right)}^{-1}
			F\left(a\right)            \\
			x\left(0\right)       & =a
		\end{aligned}
		\right.
	\end{equation}
	en dominios en los que $DF\left(x\right)$ es no singular.
	\begin{alertblock}{Observación}
		Aplicando el \alert{método de Euler} en $t=0$ con tamaño de paso
		$1$ a~\eqref{eq:ivp} desde el punto inicial $x=a$, la
		aproximación de $x\left(1\right)$ se convierte en
		\begin{equation*}
			x\left(1\right)=
			a-{\left(DF\left(a\right)\right)}^{-1}F\left(a\right)
		\end{equation*}
		que es precisamente una única iteración del método de Newton.
		Por lo tanto, la iteración de Newton puede considerarse como la
		aplicación del método de Euler con tamaño de paso $1$ en la curva
		solución dada por la homotopía de Newton~\eqref{eq:newtonhomotopy}.
	\end{alertblock}

	\begin{equation*}
		x_{n+1}=
		x_{n}-
		{\left(f^{\prime}\left(x_{n}\right)\right)}^{-1}
		f\left(x_{n}\right)
	\end{equation*}
\end{frame}