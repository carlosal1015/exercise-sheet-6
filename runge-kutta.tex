\begin{frame}
	\frametitle{Método explícito de Runge-Kutta}
	La derivación sistemática de los métodos de un paso de orden
	superior se remonta a Runge y Kutta, quienes desarrollaron métodos
	especiales de tercer y cuarto orden hace más de $122$ años.

	\begin{definition}[Runge-Kutta explícito de $m$ etapas]
		Un método se denomina \alert{Runge-Kutta explícito de $m$ etapas}
		sii la función del método $f_{h}$ es una combinación lineal de
		valores de la función $f\left(x,y\right)$ en diferentes puntos
		$\left(x,y\right)$ es
		\begin{math}
			f_{h}\left(x,y\right)=
			\gamma_{1}k_{1}\left(x,y\right)+
			\gamma_{2}k_{2}\left(x,y\right)+
			\cdots+
			\gamma_{m}k_{m}\left(x,y\right)
		\end{math}
		con
		\begin{columns}
			\begin{column}{0.48\textwidth}
				\begin{align*}
					k_{1}\left(x,y\right) & =
					f\left(
					x,y
					\right)                                 \\
					k_{2}\left(x,y\right) & =
					f\left(
					x+\alpha_{2}h,
					y+h\beta_{21}k_{1}\left(x,y\right)
					\right).                                \\
					k_{3}\left(x,y\right) & =
					f\left(
					x+\alpha_{3}h,
					y+h\left[\beta_{31}k_{1}\left(x,y\right)+\beta_{32}k_{2}\left(x,y\right)\right]
					\right)                                 \\
					                      & \vdotswithin{=} \\
					k_{m}\left(x,y\right) & =
					f\left(x+\alpha_{m}h,y+h\sum\limits_{j=1}^{m-1}\beta_{mj}k_{j}\left(x,y\right)\right),
				\end{align*}
			\end{column}
			\begin{column}{0.48\textwidth}
				\begin{equation*}
					\begin{array}
						{c|ccccc}
						0                                                                    \\
						\alpha_{2} & \beta_{21}                                              \\
						\alpha_{3} & \beta_{31} & \beta_{32}                                 \\
						\vdots     & \vdots     &            & \ddots                        \\
						\alpha_{m} & \beta_{m1} & \beta_{m2} &        & \beta_{mm-1}         \\
						\hline
						           & h_{1}      & h_{2}      & \cdots & h_{m-1}      & h_{m}
					\end{array}
				\end{equation*}
				Este método será \alert{consistente} sii
				\begin{math}
					\sum\limits_{j=1}^{m}
					h_{j}=1
				\end{math}.
			\end{column}
		\end{columns}
		donde $x=x_{j}$, $y=u_{j}$ y $h=h_{j}$ para el paso $j$.

		Por lo tanto, el método se determina fijando
		$2m-1+\dfrac{m\left(m-1\right)}{2}$ parámetros
		\begin{equation*}
			\left\{
			\gamma_{1},\dotsc,\gamma_{m},
			\alpha_{2},\alpha_{3},\dotsc,\alpha_{m},
			\beta_{21},\beta_{31},\beta_{41}\dotsc,\beta_{mm-1}
			\right\}\subset
			\mathbb{R}.
		\end{equation*}
	\end{definition}
\end{frame}

\begin{frame}
	\frametitle{Expansión de la serie de Taylor}

	La expansión de Taylor es
	\begin{equation*}
		f\left(t+\Delta t\right)=
		\sum\limits_{k=0}^{m}
		\frac{
			f^{\left(k\right)}
			\left(t\right)
		}{k!}
		\Delta t^{k}+
		\frac{
			f^{\left(m+1\right)}
			\left(\xi\right)
		}{
			\left(m+1\right)!
		}\Delta t^{m+1},\quad
		\xi\in\left[t,t+\Delta t\right].
	\end{equation*}
	Considere para $m=4$:
	\begin{equation*}
		f\left(t+\Delta t\right)=
		f\left(t\right)+
		\Delta t
		f\left(t,y\left(t\right)\right)+
		\frac{\Delta t^{2}}{2}
		f^{\left(1\right)}\left(t,y\left(t\right)\right)+
		\frac{\Delta t^{3}}{3}
		f^{\left(2\right)}\left(t,y\left(t\right)\right)+
		\frac{\Delta t^{4}}{4}
		f^{\left(3\right)}\left(t,y\left(t\right)\right)+
		O\left(\Delta t^{5}\right).
	\end{equation*}

	Considere para $m=1$:
	\begin{equation*}
		f^{\left(1\right)}\left(t,y\left(t\right)\right)=
	\end{equation*}

	% Considere para $m=2$:

	\begin{equation*}
		\systeme{
			a+b+c+d=1,
			\frac{1}{2}b+\frac{1}{2}c+d=\frac{1}{2},
			\frac{1}{4}c+\frac{1}{2}d=\frac{1}{6},
			\frac{1}{4}d=\frac{1}{24}
		}
	\end{equation*}
	que nos da $a=\frac{1}{6}$, $b=\frac{1}{3}$, $c=\frac{1}{3}$ y $d=\frac{1}{6}$.
\end{frame}

\begin{frame}
	\frametitle{Método explícito de Runge-Kutta de orden 4}
	El método clásico de Runge-Kutta es
	\begin{align*}
		k_{1}   & = f\left(x_{j},u_{j}\right).                                        \\
		k_{2}   & = f\left(x_{j}+\dfrac{h_{j}}{2},u_{j}+\dfrac{h_{j}}{2}k_{1}\right). \\
		k_{3}   & = f\left(x_{j}+\dfrac{h_{j}}{2},u_{j}+\dfrac{h_{j}}{2}k_{2}\right). \\
		k_{4}   & = f\left(x_{j}+h_{j},u_{j}+h_{j}k_{3}\right).                       \\
		u_{j+1} & = u_{j}+\frac{h_{j}}{6}\left(k_{1}+2k_{2}+2k_{3}+k_{4}\right).      \\
	\end{align*}

	Tenemos para $m=4$, $2\times 4-1+\dfrac{4\left(4-1\right)}{2}=13$
	parámetros, $11$ condiciones.

	\begin{equation*}
		% \renewcommand\arraystretch{1.3}
		\begin{array}
			{c|cccc}
			0                                                                        \\
			\dfrac{1}{2} & \dfrac{1}{2}                                              \\
			\dfrac{1}{2} & 0            & \dfrac{1}{2}                               \\
			1            & 0            & 0            & 1                           \\
			\hline
			             & \dfrac{1}{6} & \dfrac{2}{6} & \dfrac{2}{6} & \dfrac{1}{6}
		\end{array}
	\end{equation*}
\end{frame}